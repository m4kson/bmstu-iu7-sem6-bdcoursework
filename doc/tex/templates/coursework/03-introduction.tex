\chapter*{ВВЕДЕНИЕ}
\addcontentsline{toc}{chapter}{ВВЕДЕНИЕ}

Автоматизация производственных процессов на тракторном заводе имеет стратегическое значение для повышения эффективности производства и конкурентоспособности предприятия. Разработка базы данных и приложения для мониторинга поможет улучшить управление ресурсами, уменьшить затраты и увеличить производительность труда.

Разрабатываемая информационная система представляет собой комплексное решение для управления производственными процессами на тракторном заводе. Она включает в себя базу данных для хранения информации о сотрудниках, тракторах, производственных линиях, техническом обслуживании и запчастях. Приложения доступа к базе данных позволят оперативно управлять данными, отслеживать состояние оборудования, планировать и отслеживать процессы обслуживания техники.

Целью данной курсовой работы является проектирование и разработка программного обеспечения для мониторинга оборудования на тракторном заводе.

Для достижения поставленной цели необходимо решить следующие задачи:

\begin{itemize}[label=---]
    \item провести анализ предметной области и сформулировать требования и ограничения к 
разрабатываемой базе данных и приложению; 
    \item сформулировать описание пользователей проектируемого приложения;
    \item спроектировать сущности базы данных и ограничения целостности;
    \item спроектировать ролевую модель на уровне базы данных;
    \item выбрать средства реализации базы данных и приложения;
    \item описать методы тестирования разработанного функционала и разработать тесты для проверки корректности работы приложения;
    \item исследовать зависимость времени выполнения запроса от наличия индексов.
\end{itemize}